\section{Chapter5: Random variables}

\subsection{Definition}

\begin{exercise}\label{5.1.2}
Let $\{X_n\}_{n=1}^\infty$ be a sequence of random variables defined on a probability space $(\Omega, \mathcal{F}, \mathbb{P})$. For any $A \in \sigma(X_1, X_2, \dots)$ and any $\epsilon > 0$, show
that there is some $n\geq 1$ and some $B \in \sigma(X_1, \dots, X_n)$ such that $\mathbb{P}(A\Delta B) < \epsilon$.
(Hint: Use Theorem 1.2.6.)
\end{exercise}
\begin{answer} (Hao)
    We are going to show that $\mathcal{A} = \{A|\exists N\in \mathbb{N} \text{ s.t. } A \in \sigma(X_1,\dots, X_N)\}$ is an algebra that generate $\sigma(X_1,\dots)$. Then, Theorem 1.2.6 gives the desired result for the exercise.
    
     First, let's show $\mathcal{A}$ is an algebra. ($\Omega \in \mathcal{A}$): $\Omega \in \sigma(X_1) \subset \mathcal{A}$. (Closed under taking complement): $A\in \mathcal{A}$ implies $\exists n\in \mathbb{N}, A\in \sigma(X_1,\dots,X_n)$, which means $A^c \in \sigma(X_1,\dots,X_n) \subset \mathcal{A}$. (Closed under finite union): If $A_{1},\dots, A_k \in \mathcal{A}$ for some finite $k\in \mathbb{N}$, there exists $N_k \in \mathbb{N}$ s.t.  $A_1,\dots, A_k \in \sigma(X_1,\dots, X_{N_k})$. Hence, $\cup_{i=1}^k A_i \in \sigma(X_1,\dots, X_{N_k})\subset \mathcal{A}$, and we conclude that $\mathcal{A}$ is an algebra.
     
     Second, $\sigma(\mathcal{A}) = \sigma(X_1,\dots)$.
        ($\subseteq$): Since $\mathcal{A} \subset \sigma(X_1,\dots)$. ($\supseteq$): By definition in this section, $\sigma(X_1,\dots) = \sigma(\cup_{n=1}^\infty\sigma(X_n))$. Then, due to  $\sigma(X_n)\subset \sigma(X_1,\dots,X_n)\subset \mathcal{A}, n \in \mathbb{N}$, the $\sigma$-algebra generated by $\mathcal{A}$, $\sigma(\mathcal{A})$ contains $\sigma(X_1,\dots)$.
        
\end{answer}